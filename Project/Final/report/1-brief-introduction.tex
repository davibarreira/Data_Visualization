\section{A Brief Introduction}

Before delving into formal definitions, theorems and proofs, let's give an informal overview of what is
Optimal Transport, what are the main results we are interested in and how they relate to possible applications.

Optimal Transport theory main subject of study is the problem of optimally transporting
quantities from one configuration to another given a cost function. Although it may seem like a very narrow subject,
this seemly simple problem has a plethora of variations and can be significantly hard not only to solve,
but to even prove that a solution exists.

The origin of the field of Optimal Transport is usually
attributed to Gaspard Monge (1746-1818), a French mathematician, who was interested in the problem
of ``what is the optimal way to transport soil extracted from one location and move to another where it will be used,
for example, on a construction?''
\footnote{This is not a quote from Monge.}\citep{villani2008optimal}.
Monge studied this problem restricting the transportation assignment to deterministic maps, i.e. the soil
extracted from location $x$ should be moved entirely to an specific location $y$ (see Figure \ref{fig:mongeproblem}),
a condition that is known as ``non-mass splitting". Monge also considered that the cost of transportation
was proportional to the distance traveled (e.i. $c(x,y) = |x-y|$), but different cost functions can be used.

Although it has been considered the founding problem of Optimal Transport, the Monge Problem is not actually the most
common formulation when it comes to applications in Machine Learning. The formulation most used when referring
to the Optimal Transport problem is actually due to Leonid Kantorovich (1912-1986), a Russian mathematician.
Kantorovich proposed a relaxation of the non-mass splitting condition, such that the optimal transportation
solution could now transport the mass ``excavated'' from $x$ to many locations
(see Figure \ref{fig:kantorovichproblem}).

\begin{figure}[H]
  \centering
  \def\svgscale{0.7}
  \includesvg[inkscapelatex=false]{Figures/mongeproblem.svg}
  \caption{The figure illustrates the original Monge Problem, where all the mass is excavated from location
  $x$ is transported to a deterministic location $y$. The transport assignment map is represented by the arrow in green.}
  \label{fig:mongeproblem}
\end{figure}

\begin{figure}[H]
  \centering
  \includesvg[inkscapelatex=false]{Figures/kantorovichproblem.svg}
  \includegraphics[width=8cm]{Figures/kantorovichproblem.svg}
  \caption{The figure illustrates the Optimal Transport Problem with the Kantorovich relaxation.
  The transportation assignment now can split the mass in blue, transporting it to many positions.}
  \label{fig:kantorovichproblem}
\end{figure}

The transportation assignment that solves the Monge Problem is called the Optimal Transport \textbf{map},
while the solution to the Kantorovich Problem is called the Optimal Transport \textbf{plan}. As we will show
in the following sections, if the Monge Problem has a solution so does the Kantorovich Problem,
but the contrary is not always true. From here
on out, every time we refer to the OT problem, we'll be implicitly referring to the Kantorovich formulation, unless
stated otherwise.

Although the original OT problem is about soil excavation, we can apply it to abstract mathematical objects
such as probability distributions. Consider two 1-dimensional probability distributions
$\mu$ and $\nu$, and define an Optimal Transport problem where the objective is to transport
distribution $\mu$ to $\nu$ with $c(x,y) = |x-y|^p$ for $p \in [1,+\infty)$.
Note that, if the OT problem has a solution, then there exists a minimum total cost. This minimum
cost of transporting $\mu$ to $\nu$ is known as the Wasserstein distance ($W_p(\mu,\nu)$).
The use of the Wasserstein distance to measure the discrepancy between probability distributions
is one of the main applications of OT on Machine Learning.

If we want to use the Wasserstein, then many questions have to be answered:
\begin{itemize}
  \item Does the transport plan exists?
  \item If the transport plan exists, how does one obtains it and then calculates the Wasserstein distance?
  \item If the Wasserstein distance between two probability distributions goes to zero,
  does this imply convergence in probability?
\end{itemize}

The field of Optimal Transport has addressed these types of questions, thus the importance of understanding
the theory before using it on real applications.

We end this brief introduction to OT with a description of
the contents addressed in each of the following section:
\begin{enumerate}[(i)]
  \item \textbf{Monge \& Kantorovich} - We formally define the Monge Problem, the Kantorovich Problem
  and the notion of \textit{relaxation}. Then, we prove that for compact spaces with continuous cost functions,
  the Kantorovich Problem is a relaxation of the Monge Problem if the starting distribution $\mu$ is atomless;

  \item \textbf{On the Existence of Transport Plans} - This section focuses on
  proving the existence of solutions to the Optimal Transport problem.
  We first prove the existence for compact
  metric spaces with continuous cost functions, which helps us prove the more general
  existence theorem for Polish spaces with lower semi-continuous cost functions;
  
  \item \textbf{Duality Results} - The Kantorovich Problem
  admits a dual formulation, which, under some conditions, yields the same
  optimal cost as the primal formulation (i.e. strong duality). 
  This section focuses on formally introducing the dual problem and proving
  the strong duality. We start from more restricted conditions which helps us prove
  the more general cases. We finish the section with the celebrated Kantorovich-Rubinstein Duality
  Theorem, which is used in Machine Learning applications such as WGANs;

  \item \textbf{Wasserstein Distance} - We define the Wasserstein
  and show that it is formally a metric (\ref{def:metric}). Next, we prove that the convergence
  of probability measures under the Wasserstein distance is equivalent to convergence in distribution.
  We end the section with some comments on the properties of the Wasserstein distance and why
  it is useful to fields like Machine Learning.
\end{enumerate}