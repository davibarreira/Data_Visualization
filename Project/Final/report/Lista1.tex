\documentclass[10pt]{article}
\usepackage[utf8]{inputenc}
\usepackage[OT1]{fontenc}
\usepackage{amsfonts, amsmath, amsthm, amssymb}
\usepackage{bbm}
\usepackage{mathtools}
\usepackage{natbib}
\usepackage{graphicx}
\usepackage{listings}
\usepackage[margin=1in]{geometry}
\usepackage{xcolor}
\usepackage{bigints}
\usepackage{glossaries}
\usepackage{graphicx}
\usepackage{enumerate}

\theoremstyle{definition}
\newtheorem{definition}{Definition}[section]

\newtheorem{theorem}{Theorem}
\newtheorem{proposition}{Proposition}
\newtheorem{example}{Example}

\newcounter{countCode}
\lstnewenvironment{code} [1][caption=Ponme caption, label=default]{%
	\renewcommand*{\lstlistingname}{Listado} 
	\setcounter{lstlisting}{\value{countCode}} 
	\lstset{ %
	language=java,
	basicstyle=\ttfamily\footnotesize,       % the size of the fonts that are used for the code
	numbers=left,                   % where to put the line-numbers
	numberstyle=\sc,      % the size of the fonts that are used for the line-numbers
	stepnumber=1,                   % the step between two line-numbers. 
	numbersep=5pt,                 % how far the line-numbers are from the code
	numberstyle=\color{red!50!blue},
    	backgroundcolor=\color{lightgray!20},
	rulecolor=\color{blue},
	keywordstyle=\color{red}\bfseries,
	showspaces=false,               % show spaces adding particular underscores
	showstringspaces=false,         % underline spaces within strings
	showtabs=false,                 % show tabs within strings adding particular underscores
	frame=single,                   % adds a frame around the code
	framexleftmargin=0mm,
	numberblanklines=false,
	xleftmargin=5pt,
	breaklines=true,
	breakatwhitespace=true,
	breakautoindent=true,
	captionpos=t,
	texcl=true,
	tabsize=2,                      % sets default tabsize to 3 spaces
	extendedchars=true,
	inputencoding=utf8, 
	escapechar=\%,
	morekeywords={print, println, size, background, strokeWeight, fill, line, rect, ellipse, triangle, arc, save, PI, HALF_PI, QUARTER_PI, TAU, TWO_PI, width, height,},
	emph=[1]{print,println,}, emphstyle=[1]{\color{blue}}, % Mis palabras clave.
	emph=[2]{width,height,}, emphstyle=[2]{\bf\color{violet}}, % Mis palabras clave.
	emph=[3]{PI, HALF_PI, QUARTER_PI, TAU, TWO_PI}, emphstyle=[3]\color{orange!50!violet}, % Mis palabras clave.
	emph=[4]{line, rect, ellipse, triangle, arc,}, emphstyle=[4]\color{green!70!black}, % Mis palabras clave.
	%emph=[5]{size, background, strokeWeight, fill,}, emphstyle=[5]{\tt \color{red!30!blue}}, % Mis palabras clave.
	%emph={[2]sqrt,baset}, emphstyle={[2]\color{blue}}, % f(sqrt(2)), sqrt a nivel 2 se pondrá azul
	#1}}{\addtocounter{countCode}{1}}



\title{Lista de Exercícios 1}
\author{}
\date{\today}
\begin{document}
\maketitle

\noindent
\textbf{Questão 1.} Prove que se $f \in C([a,b])$, então $|f| \in C([a,b])$. Mostre que
$|f| \in C([a,b])$ não implica que $f \in C([a,b])$.
\vspace{5mm}

\noindent
\textbf{Questão 2. (Teorema do ponto fixo de Browder)} Prove que se
$f:[a,b]\to [a,b]$ é contínua, então $f$ possui um ponto fixo, ou seja, existe 
$c \in [a,b]$ tal que $f(c)=c$. Dê um exemplo de função contínua $f:[0,1)\to [0,1)$
onde não há ponto fixo.
\vspace{5mm}

\noindent
\textbf{Questão 3.} Seja $f$ contínua tal que $f(a+b)= f(a)+f(b)$ para todo
$a,b \in \mathbb R$. Prove que $f(x)=cx$, onde $c:=f(1)$.
\vspace{5mm}

\noindent
\textbf{Questão 4.} Prove que se $f_1,...,f_n$ são contínuas em $A \subset \mathbb R$,
então $h:\max\{f_1,...,f_n\}$ é contínua em $A$ (i.e. $h(x) = \max\{f_1(x),...,f_n(x)\}$).
\vspace{5mm}

\noindent
\textbf{Questão 5. (Teorema da Contração Uniforme)}
Seja $f:\mathbb R \to \mathbb R$ tal que existe $c \in (0,1)$ e
$$
|f(x) - f(y)| \leq c|x-y|, \quad \forall x,y \in \mathbb R
$$
\begin{enumerate}[(a)]
	\item Prove que $f$ é contínua.
	\item Escolha qualquer $y \in \mathbb R$ e defina uma sequência $y, f(y), f(f(y)),...$
	Prove que essa sequência converge para um ponto fixo de $f$.
	\item Prove que o ponto fixo é único.
\end{enumerate}


\nocite{*}

  \bibliography{ref}
  \bibliographystyle{plainnat}
\end{document}

